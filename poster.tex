\documentclass[14pt,landscape,color=UCLdarkred,margin=3cm]{uclposter}

\usepackage{amsmath,amssymb}
\usepackage{amsthm}
\usepackage{xcolor}
\usepackage{eso-pic}
\usepackage{authblk}
\usepackage{tikz}
\usepackage{environ}
\usepackage{graphicx}
\usepackage{float}
%\usepackage{physics}

\title{Scalable Quantum Simulation of Molecular Energies}

\author{James Mills}
\author{Shuhao Yang}
\author{Shanice St John}
\affil[1]{MSc Quantum Technologies, UCL}
% \affil[2]{TikZ, UCL}
% \affil[*]{a.example@ucl.ac.uk}

\begin{document}%\large

\maketitle

\begin{multicols}{3}

\section*{A new area for computing}


Quantum computing is a rapidly advancing field predicted to revolutionise many areas of science and technology. It is predicted to have important applications in encryption and communication, and in the development of new medicines and materials. Quantum computers are not considered to be a replacement for classical computers; they will only be useful for certain types of problems which are too difficult for classical computers to solve, such as simulating systems in quantum chemistry.

Efficient simulation of quantum chemistry experiments would enable a dramatic leap forward in our understanding of fundamental chemistry. It would significantly reduce the need for cumbersome and expensive trial-and-error techniques in the development of new medicines and materials.

\section*{Key terms}


\begin{highlightbox}[UCLdarkblue!20!white]
	\textbf{Qubit} A quantum bit of information. Its classical counterpart, the bit represents classical information and is encoded in 0s and 1s. A qubit can be represented by a state.
\end{highlightbox}

\begin{highlightbox}[UCLstone!50!white]
  \textbf{Superposition} A qubit has extraordinary property, which can be in both states of `0' and `1' at the same time, but a classical bit is either `0' or `1'.
\end{highlightbox}



\begin{highlightbox}[UCLyellow!20!white]
\textbf{Quantum simulations in chemistry} Quantum theory is our best description of phenomena happening at the smallest scales imaginable. On a quantum computer, we use quantum phenomena (the same phenomena that make the system difficult to describe with a classical computer) to precisely simulate chemistry structure and boost our computing power.
\end{highlightbox}

\begin{highlightbox}[UCLmidgreen!20!white]
\textbf{Algorithm} A set of instructions used to solve a problem, especially by a computer. The instructions are created so that it can be understood by the computer. Once sent to the quantum computer, it follows the instructions using some program or software. The end of the process provides an answer or even many possible answers!
\end{highlightbox}

\columnbreak

\section*{What are the techniques?}

The work done in the paper `Scalable Quantum Simulation of Molecular Energies' uses qubits kept at very low temperatures to run two different algorithms, the variational quantum eigensolver (VQE) and the phase estimation algorithm (PEA),  to find information about the fundamental properties of molecules. A special type of qubit known as an Xmon is used to run the algorithms. The Xmon runs using a superposition of ``charge" states.

\\
\begin{figure}[H]
  \begin{center}
%   \begin{minipage}[c]{15em}
\setlength{\fboxsep}{0.5em}
  \begin{minipage}[c]{9em}
  \begin{center}
  \includegraphics[width=7em]{4_Qubit.png}
    \caption{IBM qubit}
  \end{center}
    % \includegraphics[width=15em]{VQEdiagram.pdf}
    
  \end{minipage}
  \qquad
  \begin{minipage}[c]{20em}
%   \begin{minipage}[c]{29em}
  %\large

The qubits used to run the algorithms were a type of charge qubit that is kept at very low temperatures and is similar to this IBM qubit.

  \end{minipage}
  \end{center}

   
\end{figure}

\section*{Algorithms for Quantum Chemistry}

The VQE algorithm uses the averages of the total energy to calculate the energy of a molecule.

\begin{highlightbox}[UCLpink!20!white]
  \begin{enumerate}
\item We start by preparing a state using a ``special" method. This method makes it possible for the algorithm run in the way that we want.
\item Measure the total energy for each distance between the molecules. We must select the lowest value as we are focusing on the ground energy.
\item A tool is then used to find more lower values of the total energy.
\item Repeat until we reach the smallest value possible.
\end{enumerate}
\end{highlightbox}

The PEA algorithm estimates the value of index over exponential coefficient to calculate energy of a molecule.


\begin{highlightbox}[UCLpurple!20!white]
\begin{enumerate}
\item Similarly to the VQE algorithm, we must prepare a state using another ``special" method and encode information onto a exponential coefficient. 
\item Some equipment must be set up in a superposition of states - this will make it possible to measure the state.
\item We use operations to set up the state and the apparatus so that it is ready to be measured. 
\item Measure the state to find the lowest value of the total energy. We repeat this process until find an accurate value of the total energy.
\end{enumerate}
\end{highlightbox}



\begin{figure}[H]
  \begin{center}
   \begin{minipage}[c]{15em}
%  \begin{minipage}[c]{18em}
    \includegraphics[width=15em]{VQEdiagram.pdf}
    % \includegraphics[width=18em]{VQEdiagram.pdf}
    \caption{VQE}
  \end{minipage}
  \qquad
  \begin{minipage}[c]{15em}
%   \begin{minipage}[c]{17em}
    \includegraphics[width=15em]{PEA.pdf}
    % \includegraphics[width=17em]{PEA.pdf}
    \caption{PEA}
  \end{minipage}
  \end{center}

   
\end{figure}



\section*{Outlook}

The VQE was proven to work better than PEA because it gives more accurate results, this can be seen on Figure 4 where the VQE line is closer than the PEA line to the line representing the exact energy. This ability to simulate the quantum behaviour of atoms and molecules is hoped to offer new avenues for increasing our understanding of the world around us, and our ability to develop new medicines and materials!

\begin{figure}[H]
  \begin{center}
  \includegraphics[scale=1.2]{result.pdf}
  \caption{Computed $H_2$ energy curve, energy surface of molecular hydrogen as determined by both VQE and PEA}
  \end{center}
    
 

   
\end{figure}



\end{multicols}
	
\end{document}
