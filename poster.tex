\documentclass[14pt,landscape,color=UCLdarkred,margin=3cm]{uclposter}

\usepackage{amsmath,amssymb}
\usepackage{amsthm}
\usepackage{xcolor}
\usepackage{eso-pic}
\usepackage{authblk}
\usepackage{tikz}
\usepackage{environ}
%\usepackage{ulgothic}
\usepackage{graphicx}
\usepackage{float}
\usepackage{physics}

\title{Scalable Quantum Simulation of Molecular Energies}

\author[1 *]{Shuhao Yang}
\author[1,2]{James Mills}
\author[1,2]{Shanice St John}
\affil[1]{Department of LaTeX Studies, UCL}
\affil[2]{TikZ, UCL}
\affil[*]{a.example@ucl.ac.uk}

\begin{document}

\maketitle

\begin{multicols}{3}

\section*{Introduction}
Quantum computing is a rapidly advancing field predicted to revolutionise many
areas of science and technology. It is predicted to have important real-world applications in encryption and communication systems and in the development of new medicines and materials to name but a few. An important point is that these quantum computers are not considered a replacement to classical computers, they will only be brought to bear on certain types of problems too difficult for classical computers.

Simulating systems in quantum chemistry is one of those problems too difficult for classical computers. If we can efficiently simulate quantum chemistry experiments it would enable a dramatic leap forward in our understanding of fundamental chemistry, and be hugely impactful to a number of fields of endeavour. For example it would significantly reduce the need for cumbersome and expensive trial-and-error techniques in the development of new medicines and materials.

\section*{Installation and usage}

Copy the `uclposter.sty` file to the directory containing the poster LaTeX file.
No additional files are needed (the banner/logo is included entirely as vector drawing code).

\section*{Highlight boxes}

\begin{highlightbox}
	The \textbackslash highlightbox command inserts a coloured box, the default colour is a lighter version of the banner colour.
\end{highlightbox}

\begin{highlightbox}[UCLdarkblue!20!white]
	The colour of a \textbackslash highlightbox can also be changed.
\end{highlightbox}

\begin{highlightbox}[UCLdarkblue!20!white]
	Qubit
\end{highlightbox}

\begin{highlightbox}[UCLdarkblue!20!white]
	Qubit
\end{highlightbox}

\begin{highlightbox}[UCLdarkblue!20!white]
	Qubit
\end{highlightbox}

\columnbreak

\section*{Techniques in Paper}



\begin{figure}[H]
  \begin{center}
  \begin{minipage}[c]{12em}
    
    \includegraphics[width=10em]{VQEdiagram.pdf}
    \caption{VQE}
  \end{minipage}
  \qquad
  \begin{minipage}[c]{12em}
    \centering
    \includegraphics[width=10em]{PEA.pdf}
    \caption{PEA}
  \end{minipage}
  \end{center}

   
\end{figure}



\columnbreak

\section*{Conclusion and Outlook}

\begin{highlightbox}[UCLdarkblue!20!white]
	The colour of a \textbackslash highlightbox can also be changed.
\end{highlightbox}



\end{multicols}
	
\end{document}
